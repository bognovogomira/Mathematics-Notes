\documentclass[9pt]{report}

\usepackage{amssymb, amsmath}
\usepackage{thmbox}
\usepackage{bbold}
\usepackage{enumerate}
\usepackage{setspace}
\usepackage[light,math]{iwona}
\usepackage[T1]{fontenc}

\newlength{\tindent}
\setlength{\tindent}{\parindent}
\setlength{\parindent}{0em}
\setlength{\parskip}{0.8em}
\onehalfspacing

\makeatletter
\renewcommand\thmbox@headstyle[2]{{\scshape #1 #2 }}
\renewcommand\thmbox@titlestyle[1]{ \bfseries \boldmath(#1)}
\makeatother

\newtheorem[L]{theorem}{Theorem}[chapter]
\newtheorem{lemma}[theorem]{Lemma}
\newtheorem{corollary}[theorem]{Corollary}

\makeatletter
\renewcommand\thmbox@headstyle[2]{{\scshape #1 #2 }}
\renewcommand\thmbox@titlestyle[1]{ \bfseries \boldmath(#1)}
\renewcommand\thmbox@bodystyle{}
\makeatother


\newtheorem[L,nocut]{definition}[theorem]{Definition}

\makeatletter
\renewcommand\thmbox@headstyle[2]{{\scshape #1}}
\renewcommand\thmbox@titlestyle[1]{ \bfseries \boldmath(#1)}
\renewcommand\thmbox@bodystyle{}
\makeatother

\newtheorem[S,nounderline]{remark}[theorem]{Remarks}

\newenvironment{Proof}{ 
	\\ \setlength{\parindent}{\tindent}
\begin{proof}}
{\end{proof} \setlength{\parindent}{0pt}}

\newenvironment{Example}{ 
	\setlength{\parindent}{\tindent}
\begin{example}}
{\end{example} \setlength{\parindent}{0pt}}

\begin{document}

\begin{titlepage}
    \begin{center}
    \end{center}
\end{titlepage}

\chapter{Algebras and Measures}

Does every subset of \(\mathbb{R}^n\) have a `volume'? What we require is a
non-negative valued function on the powerset \(2^{(\mathbb{R}^n)}\) that
behaves nicely like a volume-measure should. Sadly, this is not possible
(as we will see), and best we can do is define such a volume-measure on
special subsets of \(\mathbb{R}^n\). More generally, our question is about
what volume-measures we are allowed to define, and how they behave.

Since we wish to study real-valued functions on subsets of a space, the following definition gives names to desirable properties:

\begin{definition}[Set functions] 
	For \(\mathcal{A}\subset 2^A\) with \(\emptyset \in \mathcal{A}\), 
	\(\mu: \mathcal{A}\rightarrow [0,+\infty]\) is a set function on
	\(A\) if \(\mu(\emptyset) = 0\). Additionally, we say \(\mu\) is 
	\begin{enumerate}[(i)]
		\item finitely (resp. countably) additive, if for finite
			(resp. countable) collections \(\{A_i\}\subset
			\mathcal{A}\) of disjoint sets, \(\mu(\bigsqcup
			A_i) = \sum \mu(A_i)\) whenever defined.
		\item finitely (resp. countably) subadditive, if for finite
			(resp. countable) collections \(\{A_i\}\subset
			\mathcal{A}\) of sets, \(\mu(\bigcup A_i)\leq \sum
			\mu(A_i)\) whenever defined.
		\item increasing, if \(A\subset B \; \Rightarrow \;
			\mu(A)\leq \mu(B)\). 
	\end{enumerate} 
\end{definition}

\section{Collections of subsets}

We need collections of subsets with desirable closure properties to define our set-functions on. Such collections form an algebra.

\subsection{Boolean algebras and Means} 

When measuring volumes in \(\mathbb{R}^3\), it is natural to consider all subsets that can be written as finite unions of boxes, since the volumes of those are easiest to define. Algebras like these are nice to work with.

\begin{definition}[Boolean algebra]
	A boolean algebra on \(A\) is a subset of \(2^A\) containing
	\(\emptyset\) that is closed under taking complements and finite
	unions.
\end{definition}

It is easy to see that Boolean algebras are closed under taking finite intersections, set-differences and symmetric differences.

For \(A\) a set and \(\mathcal{A} \subset 2^A\), \emph{the boolean algebra generated by \(\mathcal{A}\)} is the smallest boolean algebra \(\text{b}(\mathcal{A})\) containing \(\mathcal{A}\), i.e.\ the intersection of all boolean algebras containing \(\mathcal{A}\). This admits a nice description- \(\text{b}(\mathcal{A})\) is the collection of all finite unions and intersections of elements of \(\mathcal{A}\) and their complements.

For instance, the boolean algebra generated by the topology on a set is the \emph{algebra of constructible sets}, i.e. sets that are locally closed (closed in an open subspace, i.e. of the form \(\text{open} \cap \text{closed}\)).

When considering set functions on Boolean algebras, it is natural to consider those that respect finite unions:

\begin{definition}[Mean]
	For \(A\) a set and \(\mathcal{A}\subset 2^A\) a boolean algebra, a \emph{mean} on \((A,\mathcal{A})\) is a finitely additive set function \(m\) on \(A\).
\end{definition}

\begin{remark}
	A mean is always finitely subadditive and increasing.
\end{remark}

Then we can measure the volumes of a large chunk of subsets of \(\mathbb{R}^n\) by considering the natural mean \(m\) on the boolean algebra generated by boxes. Call such a set \emph{elementary}. We can in fact extend \(m\) to more than just elementary sets, by allowing things that they limit to: a subset \(A\subset \mathbb{R}^n\) is \emph{Jordan-measurable} if for any \(\epsilon>0\) there are elementary sets \(E \subset A \subset F\) such that \(m(F\setminus E) < \epsilon\). The collection of all Jordan-measurable subsets is again a boolean algebra, and \(m\) is a mean on it.

\begin{remark}
	The boolean algebra of Jordan-measurable sets is genuinely bigger than the one of elementary sets. For instance, the \(n\)-sphere is not a finite union of boxes, but is still Jordan-measurable. 
\end{remark}

The Jordan measure is reminiscent of the Riemann integral, and the relation is in fact very strong: a subset of \(\mathbb{R}\) is Jordan-measurable if and only if its indicator function is Riemann integrable. With the similarities, the Jordan measure also inherits the shortcomings of the Riemann integral.

\begin{Example}
\(\mathbb{Q}\cap [0,1]\) is not Jordan-measurable, for any elementary set containing it must have mean atleast \(1\) while any elementary set inside it must be finite i.e.\ of zero mean.
\end{Example}

There are also sets that are not Jordan-measurable, but have volumes in the traditional sense. In particular, Jordan-measurable sets are always bounded so something like Gabriel's horn (which has a finite volume in \(\mathbb{R}^3\)) is not Jordan-measurable.

\subsection{Limiting behaviour}

One shortcoming boolean algebras have is that they are restricted to finite unions, whereas when thinking of volumes it is natural to consider limits of series: \(1 + \frac{1}{2} + \frac{1}{4} + ... = 2\). In the transition from elementary sets to Jordan-measurable ones, we essentially allowed only \emph{some} countable unions. Thus it makes sense to extend our definitions to \emph{all} countable unions, and talk about set functions that respect limiting behaviour:

If \(A_1\subset A_2 \subset ... \subset A\) is a nested sequence of sets such that \(\bigcup_n A_n = A\), we write \(A_n \uparrow A\). Similarly for nested sequences \(A_1\supset A_2 \supset ... \supset A\) with \(\bigcap_nA_n=A\), we write \(A_n \downarrow A\).

\begin{definition}
	For a set \(A\), a set function \(\mu\) on \(A\) is said to be 
	\begin{enumerate}[(i)]
		\item \emph{upward monotone convergent}, if for increasing sequences \(B_n \uparrow B\) we have \(\mu(B_n)\rightarrow \mu(B)=\sup_n\mu(B_n)\) whenever defined.
		\item \emph{downward monotone convergent}, if for decreasing sequences of \(\mu\)-finite sets \(B_n\downarrow B\), we have \(\mu(B_n)\rightarrow \mu(B)=\inf_n \mu(B_n)\) whenever defined.
	\end{enumerate}
\end{definition}

\begin{remark}
	We only consider \(\mu\)-finite sequences when talking of downward monotone convergence to get around trivial cases like \([n,\infty)\) where a sequence of large sets has empty intersection.
\end{remark}

\textbf{(Aside)} Why only countable, and not arbitrary?	As it turns out, countability is the furthest it makes sense to go, since the sum of uncountably many positive reals cannot be finite:

\begin{theorem}
	If we have a collection \(n_i \geq 0\) of reals with \(\sum_{i \in I}n_i < \infty\) then at most countably many \(n_i\) can be non-zero.
	\begin{Proof}
		Let \(S_n = \{n_i | n_i > \frac{1}{n}\}\), for \(n \in \mathbb{N}\). If any \(S_n\) has infinitely many elements, then \(\sum_{i \in I}n_i > \sum_{m \in S_n} m= \infty\), hence each \(S_n\) is finite. Then the set of non-zero elements is \(\bigcup_{n\in \mathbb{N}}S_n\) which is a countable union of finite sets.
	\end{Proof}
\end{theorem}

We can consider algebras which include arbitrary unions, but then 'most' of the elements in the algebra would have infinite (or zero) volume and hence the information we get would be largely uninteresting. 

\subsection{Sigma algebras and Measures}

The previous discussion motivates us to consider extensions of Boolean algebras that are closed under countable unions:

\begin{definition}[\(\sigma\)-algebra]
A \(\sigma\)-algebra \(\mathcal{A}\) on \(A\) is a subset of \(2^A\) that contains \(\emptyset\) and is closed under taking complements and countable unions. The pair \((A,\mathcal{A})\) is called a \emph{measurable space}, and each element of \(\mathcal{A}\) is a measurable set.
\end{definition}

\(\sigma\)-algebras are automatically closed under taking countable intersections, set differences and symmetric differences.

For \(A\) a set and \(\mathcal{A}\subset 2^A\), the \emph{\(\sigma\)-algebra generated by \(\mathcal{A}\)} is the smallest \(\sigma\)-algebra \(\sigma(\mathcal{A})\) containing \(\mathcal{A}\), i.e. the intersection of all \(\sigma\)-algebras containing \(\mathcal{A}\). Unlike boolean algebras these admit no simple description, since taking countable unions and complements is not a straightforward procedure.

\begin{Example}
	The \(\sigma\)-algebra generated by the topology on a Polish (separable and metrizable) space \(X\) is called the \emph{Borel \(\sigma\)-algebra} \(\mathcal{B}(X)\). It is easy to see that this exists because the powerset itself is a \(\sigma\)-algebra containing the topology, but constructing it requires using transfinite induction to generate the \emph{Borel Heirarchy}: 
	\begin{itemize}
		\item \(\mathbf{\Sigma^0_1}\) is the collection of all open sets, \(\mathbf{\Pi}^0_1\) is the collection of all closed sets.
		\item \(\mathbf{\Sigma}^0_\alpha\) for \(\alpha > 1\) is generated by taking countable unions of elements in \(\bigcup_{\beta < \alpha} \mathbf{\Pi}^0_\beta\). (Sets of this class are closed under countable unions.)
		\item A set is in \(\mathbf{\Pi}^0_\alpha\) iff its complement is in \(\mathbf{\Sigma}^0_\alpha\). (Sets of this class are closed under countable intersections)
		\item \(\mathbf{\Delta}^0_{\alpha} = \mathbf{\Sigma}^0_\alpha \cap \mathbf{\Pi}^0_\alpha\)
	\end{itemize} 
        Then \(\mathbf{\Delta}^0_{\omega_1}\) is the Borel \(\sigma\)-algebra on \(A\). 
\end{Example}

Set functions on \(\sigma\)-algebras occupy a central role in this topic, and are defined to respect countable unions.

\begin{definition}[Measure]
A \emph{measure} on \((A,\mathcal{A})\) is a countably additive set function \(\mu\) on \(A\).
The triple \((A,\mathcal{A},\mu)\) is called a \emph{measure space}. 
\end{definition}

\begin{remark}
	A measure is always countably subadditive and increasing, and has upward and downward monotone convergence.
\end{remark}

\section{Specifying the measure}

Since \(\sigma\)-algebras cannot be explicitly written down in non-trivial cases, specifying the measure on each measurable set is impossible. What we hope to do instead, is specify its value on a smaller collection of sets in a way that it can be extended to the whole \(\sigma\)-algebra in a unique way. Any boolean subalgebra works for instance, but that is overkill. This raises two questions: 
\begin{enumerate}[(i)]
	\item How much information is needed to identify a measure uniquely?
	\item How much information is needed to \emph{construct} the measure in a usable form?
\end{enumerate}
As it turns out, the two answers are different, and we start by answering the former. 
 
\subsection{Uniqueness of extension}
 
Trivially, you can conclude that two measures are the same if their actions coincide on the entire \(\sigma\)-algebra. However, this is unnecessary: if you know two measures coincide on a countable collection of disjoint sets then they must coincide on the union. Similarly if you know two measures agree on \(X\) and \(Y \supset X\) then they must agree on \(Y\setminus X\). Call the latter property \emph{closure under local complements}\footnote{Non-standard terminology}. 

\begin{definition}[\(d\)-systems]
	\(\mathcal{A} \subset 2^A\) is a \(d\)-system (or \emph{Dynkin system}) if it contains \(A\) and is closed under taking local complements and countable disjoint unions.
\end{definition}

It is easily seen that \(d\)-systems are closed under taking complements. They are also closed under taking unions of upwards-nested sequences, i.e. if \(X_1 \subset X_2 \subset X_3 \subset ...\) are in \(\mathcal{A}\) then so is \(\bigcup X_i\).  

The upward-nested restatement is very nice because we can say a \(d\)-system is closed under countable unions if and only if it is closed under finite unions (or finite intersections, since we have complements.) In particular we have this: 

\begin{lemma}[\(\pi\)-closures of \(d\)-systems]
	A family of subsets is a \(\sigma\)-algebra if and only if it is a \(d\)-system closed under finite intersections. 
\end{lemma}

Checking two measures coincide on intersections is necessary because knowing the measures of two sets does not uniquely determine the measure of their intersections. 

\begin{definition}[\(\pi\)-systems]
	\(\mathcal{A} \subset 2^A\) is a \emph{\(\pi\)-system} of it contains \(\emptyset\) and is closed under finite intersections.
\end{definition}

We just saw that \(\pi\)-closures of \(d\)-systems are always \(\sigma\)-algebras. The converse is true as well:

\begin{lemma}[Dynkin's]
	If \(\mathcal{A}\) is a \(\pi\)-system on \(A\), then any \(d\)-system containing \(\mathcal{A}\) also contains \(\sigma(\mathcal{A})\). 
	\begin{Proof}
		We show that \(\mathcal{D}=\bigcap \{d\text{-systems containing }\mathcal{A}\}\) is itself a \(\sigma\)-algebra. It is clearly a \(d\)-system, and contains \(\mathcal{A}\). To show it is closed under finite intersections, consider \(\mathcal{D}' = \{D \in \mathcal{D}\;|\; D \cap X \in \mathcal{D} \quad \forall X \in \mathcal{D}\}\). This contains \(\mathcal{A}\):\\ 

		\noindent Look at \(\mathcal{D}'' = \{D \in \mathcal{D} \; | \; D \cap X \in \mathcal{D} \quad \forall X \in \mathcal{D}\}\). This is a \(d\)-system, and \(\mathcal{A}\) is a \(\pi\)-system so lies within \(\mathcal{D}''\). Then by the minimality of \(\mathcal{D}\), we have \(\mathcal{D}' = \mathcal{D}\). \\ 

		\noindent But \(\mathcal{D}'\) is also a \(d\)-system, so by the minimality of \(\mathcal{D}\) we have \(\mathcal{D}'' = \mathcal{D}\) which is hence closed under finite intersections.
	\end{Proof}
\end{lemma}

This sets us up for the main result, that knowing the values of a measure on a \(\pi\)-system is \emph{sufficient} when the space isn't too large (i.e. the measure is finite-valued). 
 
\begin{theorem}
	Suppose \(\mathcal{A}\subset 2^A\) is a \(\pi\)-system on \(A\), and \(\mu_1, \mu_2\) are two finite-valued measures on \((A,\sigma(\mathcal{A}))\) that agree when restricted to \(\mathcal{A}\cup \{A\}\). Then they are equal.
	\begin{Proof}
		Let \(\mathcal{D}=\{X\subset \sigma(\mathcal{A}) \; | \; \mu_1(X) = \mu_2 (X)\}\supset \mathcal{A} \cup \{A\}\). If we show \(\mathcal{D}\) is a \(d\)-system, we are done by the lemma. If \(B,C \in \mathcal{D}\) with \(B\subset C\), then since measures are additive and finite-valued we have \(\mu_1(C\setminus B) = \mu_1(C)-\mu_1(B) = \mu_2(C)-\mu_2(B) = \mu_2(C\setminus B)\). Similarly, if \(A_1, A_2,...\) are disjoint sets in \(\mathcal{D}\) then \(\mu_1(\bigsqcup_1^\infty A_i) = \lim_{n\rightarrow \infty} \sum_1^n \mu_1(A_i) =\lim_{n\rightarrow \infty} \sum_1^n \mu_2(A_i) = \mu_2(\bigsqcup_1^\infty A_i) \).
	\end{Proof}
\end{theorem}

\subsection{Constructing the extension} 

While \(\pi\)-systems uniquely determine the measure on the rest of the space, that usually isn't enough information to readily study the resulting measure. To define an extension constructively, we demand slightly more structure:

\begin{definition}[Ring]
	A ring \(\mathcal{A}\) on \(A\) is a subset of \(2^A\) containing \(\emptyset\) that is closed under finite unions and set-difference.
\end{definition}

\begin{remark}
	Rings are closed under finite intersection and symmetric differences. \\ \\ 
	\((\mathcal{A}, \Delta, \cap)\) is an algebraic ring with zero \(\emptyset\). Boolean algebras then correspond to rings with a unit (the set \(A\)).
\end{remark}

It is clear that every boolean algebra is a ring, but a ring need not contain the whole set. Intuitively, a ring cannot 'climb' to the topmost level (with just finite unions) while a boolean algebra can. For instance, all finite subsets of \(\mathbb{N}\) form a ring which isn't a boolean algebra. Thus we are still demanding less information than a boolean algebra. We already know that knowing how a measure behaves on a generating ring (which is a \(\pi\)-system) is enough to identify it uniquely; we claim that the information is also is enough to \emph{construct} the rest of the measure.

\begin{theorem}[Caratheodory's extension]
	Suppose \(\mathcal{A}\) is a ring on \(A\) and \(\mu: \mathcal{A}\rightarrow[0,\infty]\) is a countably additive set-function. Then \(\mu\) can be extended to a measure on \(\sigma(\mathcal{A})\).
	\begin{Proof}
		We define the \emph{outer measure} \(\mu^*: 2^A\rightarrow [0,\infty] \) by 
		\[\mu^*(X) = \inf\left\{\sum_1^\infty \mu(A_i) \; \middle| \quad A_i \in \mathcal{A}, \; X \subset \bigcup_1^\infty A_i\right\}\]
		(assuming \(\inf \emptyset = \infty\)). This is the required extension. 
	\end{Proof}
\end{theorem} 

\begin{remark}
    Countably additive set-functions on rings are termed \emph{pre-measures}.
\end{remark}

In what follows, we prove that Caratheodory's outer measure is indeed a measure on \emph{some} \(\sigma\)-algebra containing \(\mathcal{A}\). 

\begin{lemma}[Countable subadditivity of \(\mu^*\)]
	Caratheodory's outer measure is countably subadditive.
	\begin{Proof}
		Suppose \(X_1,X_2,...\) is a countable sequence in \(2^A\). If \(\mu^*(\bigcup_1^\infty X_i) < \infty\), then for any \(\epsilon>0\) and \(X_n\) there is an \(\mathcal{A}\)-cover \(A_{n1},A_{n2},...\) such that \(\mu^*(X_n) + \frac{\epsilon}{2^n} \geq \sum_{i=1}^\infty \mu(A_{ni})\). Then \(A_{mn}\) is an \(\mathcal{A}\)-cover of \(\bigcup_1^\infty X_i\), so we have \(\mu^*(\bigcup_1^\infty X_i) \leq \sum_{j=1}^\infty \sum_{i=1}^\infty A_{ji} \leq \epsilon + \sum_1^\infty X_i\). Since \(\epsilon>0\) was arbitrary, we get the required inequality.
	\end{Proof}
\end{lemma} 

At this point it is imperative we check \(\mu^*\) is indeed an \emph{extension} of \(\mu\) as claimed:

\begin{lemma}
	\(\mu^*\) and \(\mu\) agree on \(\mathcal{A}\)
	\begin{Proof}
		Suppose \(X\in \mathcal{A}\). Then for any \(\mathcal{A}\)-cover \(A_1,A_2,...\) of \(X\), countable subadditivity of \(\mu\) gives us \(\mu(X) \leq \sum_1^n\mu(A_n)\). Since this is true for all covers, it must be true for the infimum over all covers i.e. \(\mu(X)\leq \mu^*(X)\). But one of these covers is \(X,\emptyset, \emptyset, ...\) so \(\mu^*(X)\leq \mu(X)\). 
	\end{Proof}
\end{lemma}

We now define a domain for \(\mu^*\), the \(\sigma\)-algebra of outer-measurable sets. 

\begin{definition}[The Caratheodory condition]
	We say \(M \in 2^A\) is \emph{\(\mu^*\)-measurable} if \(\forall X \in 2^A, \; \mu^*(X) = \mu^*(X\cap M) + \mu^*(X \cap M^c)\).
\end{definition}

Let \(\mathcal{M}\) be the collection of all such sets, then \(\mathcal{A}\subset \mathcal{M}\).
		Indeed, if \(A_0 \in \mathcal{A}\) and \(X\in 2^A\), then subadditivity gives us \(\mu^*(X) \leq \mu^*(X\cap A_0) + \mu^*(X\cap A_0^c)\). On the other hand, wlog \(\mu^*(X)<\infty\) and hence for any \(\epsilon>0\) there is an \(\mathcal{A}\)-cover \(A_1,A_2,...\) such that \(\mu^*(X) +\epsilon \geq \sum_1^\infty \mu(A_i)\). Then we have \(\mathcal{A}\)-covers given as \(X\cap A_0 \subset \bigcup_1^\infty (A_0\cap A_i)\), \(X \cap A_0^c \subset \bigcup_1^\infty (A_0^c \cap A_i)\). This immediately gives us 
		\begin{align*}
			\mu^*(X\cap A_0) + \mu^*(X\cap A_0^c) 
			&\leq \sum_1^\infty \mu(A_i\cap A_0) + \sum_1^\infty\mu(A_i\cap A_0^c) \\
			&= \sum_1^\infty \mu(A_i) \\
			&\leq \mu^*(X) + \epsilon
		\end{align*}
		Since this holds for arbitrary \(\epsilon\), \(A_0\) is \(\mu^*\)-measurable.

We are in good shape to conclude the proof.

\begin{lemma}
	The collection \(\mathcal{M}\) of outer-measurable sets is a \(\sigma\)-algebra, and \(\mu^*\) is a measure on \(\mathcal{M}\).
	\begin{Proof}
		To see \(\mathcal{M}\) is a \(\pi\)-system, observe that \(\emptyset\in \mathcal{M}\) and if \(M,N\in \mathcal{M}\) then for any \(X \in 2^A\) we have 
		\begin{align*}
			\mu^*(X) &= \mu^*(X \cap M) + \mu^*(X \cap M^c) \\ 
				 &= \mu^*(X\cap M \cap N) + \mu^*(X \cap M \cap N^c) + \mu^*(X \cap M^c )\\
				 &= \mu^*(X\cap M \cap N) + \mu^*(X \cap M \cap N^c) + \mu^*(X \cap (M \cap N)^c \cap M^c )\\
				 &= \mu^*(X\cap (M \cap N)) + \mu^* (X \cap (M\cap N)^c \cap M) + \mu^*(X \cap (M \cap N)^c \cap M^c) \\ 
				 &= \mu^*(X\cap (M\cap N)) + \mu^* (X\cap (M\cap N)^c)
		\end{align*}
		Hence \(\mathcal{M}\) is closed under taking finite intersections. \\ \\ 
		To see it is a \(d\)-system, see that is closed under taking complements from definition. Now if \(M_1, M_2, ...\) is a collection of disjoint \(\mu^*\)-measurable sets then for any \(X\in 2^A\), we have 
		\begin{align*}
			\mu^*(X)&=\mu^*(X \cap M_1) + \mu^* (X \cap M_1^c) \\ 
				&= \mu^*(X\cap M_1) + \mu^* (X\cap \underbrace{M_1^c \cap M_2}_{M_2}) + \mu^* \left(X \cap M_1 ^c \cap M_2^c\right) \\ 
				&\vdots \\
				&= \sum_1^n \mu^*\left(X\cap M_i\right) + \mu^* \left(X \cap \left(\bigsqcup_1^n M_i\right)^c\right)
		\end{align*}
		for all finite \(n\). Since \(\mu^*\) is increasing we have  \(\mu^*(X\cap (\bigsqcup_1^nM_i)^c)\geq \mu^*(X\cap (\bigcup_1^\infty M_i)^c)\), i.e. \(\mu^*(X)\geq \sum_1^n \mu^*(X\cap M_i) + \mu^* (X\cap (\bigsqcup_1^\infty M_i)^c)\). This must also hold in the limit, and then using countable subadditivity we get 
		\begin{align*}
			\mu^*(X)&\geq \sum_1^\infty \mu^*(X\cap M_i) + \mu^*\left(X\cap \left(\bigsqcup_1^\infty M_i\right)^c\right) \\ 
				&\geq \mu^*\left(X\cap \bigsqcup_1^\infty M_i\right) + \mu^*\left(X\cap \left(\bigsqcup_1^\infty M_i\right)^c\right).
		\end{align*}
		On the other hand, subadditivity gives us 
		\[\mu^*\left(X\cap \bigsqcup_1^\infty M_i\right) + \mu^*\left(X \cap \left(\bigsqcup_1^\infty M_i\right)^c\right) \leq \mu^*(X).\] 
		This simultaneously shows us \(\mu^*\) is countably additive, and hence a measure on \(\mathcal{M}\). 
	\end{Proof}
\end{lemma}

Since \(\mu\) extends to a measure on atleast one \(\sigma\)-algebra containing \(\mathcal{A}\), it extends to the smallest one, \(\sigma(\mathcal{A})\).

\subsubsection{Complete algebras and the Caratheodory condition}
NOTE In fact, we extended \(\mu\) to a \(\sigma\)-algebra /larger/ than \(\sigma(\mathcal{A})\). How much larger?

    DEFINITION If \((A,\mathcal{A},\mu)\) is a measure space, call a subset \(X\in 2^A\) /null/ if \(\exists \bar{X}\in \mathcal{A}\) such that \(\mu(\bar{X})=0, \; X \subset \bar{X}\).

    Then it can be shown that if \((A,\mathcal{A},\mu)\) is a measure space and \(\mu^*\) is the corresponding outer measure constructed as in Caratheodory's proof, the set of \(\mu^*\)-measurable sets is exactly of the form \(\{X\cup N \; | \; X \in \mathcal{A}, \; N \text{ null} \} \). This is called the /completion/ of \(\mathcal{A}\) (see exercise sheet 1.) Note that taking the completion of an algebra is an idempotent operation.

\section{The Lebesgue Measure} 

Given a pre-measure \(\mu\) on a set \(A\), the extended measure \(\mu^*\) given by Caratheodory is unique if it turns out that \(\mu(A)<\infty\). We can in fact do better, and allow cases where \(A\) is only 'countably finite': 

\begin{definition}[\(\sigma\)-finiteness]
	A set-function \(\mu\) on \(A\) is \(\sigma\)-finite if \(A\) can be covered by countably many \(\mu\)-finite sets. 
\end{definition}

\begin{remark}
	This helps avoid edge cases like for example the ring on \(\mathbb{R}\) generated by semi-open intervals \((a,b]\) where each interval has pre-measure \(\infty\). This can be extended in two ways- either let everything non-empty in the \(\sigma\)-algebra go to \(\infty\), or simply the counting measure (where singleton sets have measure \(1\)). Note that the second one doesn't arise from Caratheodory.
\end{remark}

It can easily be seen that the extension given by Caratheodory's theorem is unique if the pre-measure is \(\sigma\)-finite. We can then apply Caratheodory's construction to the Jordan mean to get a measure on \(\mathbb{R}\).

\begin{theorem}[Lebesgue's measure]
	There exists a unique translation-invariant Borel measure \(\mu\) (called the \emph{Lebesgue measure}) on \(\mathbb{R}\) such that \(\mu (0,1] = 1\). 
	\begin{Proof}
		Such a measure \(\mu\), if it exists, must have the property that for \(a_1<b_1<a_2<...<b_n\), \(\mu(\bigsqcup_1^n (a_i,b_i]) = \sum_1^n (b_i-a_i)\). To see this, by translation-invariance and additivity it suffices to show \(\mu (0,a] = a\). If \(a \in \mathbb{N}\), this is clear. If \(a \in \mathbb{Q}_{\geq 0}\), then \(a b \in \mathbb{N}\) for some \(b\) and hence \(\bigsqcup_{i=0}^{b-1}(0+i a,a + i a] = (0,a b]\). If \(a\) is a positive irrational number, then there is an increasing sequence \(a_n \rightarrow a\) of rationals. The increasing sequence of intervals \((0,a_n]\) converges to \((0,a]\), and hence by convergence properties of the measure it can be shown that \(a_n = \mu (0,a_n] \rightarrow \mu (0,a]\). But limits are unique so \(\mu(0,a]=a\). \\ \\ 
		Consider the ring on \(\mathbb{R}\) of semi-open intervals \(\mathcal{A}=\{\bigsqcup_1^n (a_i,b_i] \, |\, b_n>a_n>\dots>b_1>a_1 \}\). This generates \(\mathcal{B}\) the Borel \(\sigma\)-algebra, and has a pre-measure given by \(\mu(\bigsqcup_1^n (a_i,b_i]) = \sum_1^n(b_i-a_i)\). This is well-defined and countably additive: if \(\bigsqcup_1^n C_i = \bigsqcup_1^m D_i\), then we have \(\bigsqcup_{i=1}^n C_i = \bigsqcup_{i=1}^n \bigsqcup_{j=1}^m C_i\cap D_j = \sum_{i=1}^n \sum_{j=1}^m \mu(C_i\cap D_j)\) and the result follows from symmetry. Since \(\mu\) is clearly additive, we only need to show it is also continuous: suppose \(A_n \downarrow \emptyset\) is a sequence in the ring. If \(\mu(A_n)\nrightarrow 0\), then there is an \(\epsilon\) such that \(\forall n, \; \mu(A_n) > 2\epsilon\). Then for each \(\epsilon\) we can find a \(B_n \in \mathcal{A}\) such that \(\bar{B_n}\subset A_n\) with \(\mu(A_n \setminus B_n)\leq 2^{-n}\epsilon\). (The \(B_n\)s form tighter and tighter approximations to \(A_n\) from within). By construction, we have
  
         \[ \mu(A_n \setminus \bigcap_{i=1}^n B_i) = \mu(\bigcup_{i=1}^n A_n\setminus B_i) \leq \mu(\bigcup_{i=1}^n A_i\setminus B_i) \leq \sum_{i=1}^n\mu(A_i\setminus B_i) \leq \epsilon. \]

        But this means \(\mu(\bigcap_1^n B_i)\geq \epsilon\), so \(\bigcap_1^n B_i\) is non-empty for each \(n\). Let \(K_n = \bigcap_1^n \bar{B_i}\), this is non-empty and compact. But \(K_1,K_2,...\) is a decreasing sequence of nested non-empty compact sets, hence the topology on \(\mathbb{R}\) dictates that \(\bigcap_1^\infty K_i\) is non-empty. Thus we have \(\emptyset \neq \bigcap_1^\infty K_i \subset \bigcap_1^\infty A_i = \emptyset\), a contradiction.

    Moreover, \(\mathbb{R}\) is \(\sigma\)-finite so by Caratheodory's theorem, we know that the pre-measure extends to a unique such measure. \(\square\)
    \end{Proof}
    \end{theorem}
    
    /The hard part of the proof is showing countable additivity of the pre-measure, i.e. that any sequence \(A_n \downarrow \emptyset\) must have measure \(0\) in the limit. Because we are dealing with semi-open intervals, there isn't a direct approach and we have to instead reason about contained closed sets that remain 'large enough' (measure-wise, and then use the fact that non-zero measure sets cannot be empty). The transition to compact sets allows us to use the powerful nested intervals property./

The completion of \(\mathcal{B}\) (obtained by allowing unions with null sets) is the class of /Lebesgue-measurable/ sets. Is every set Lebesgue measurable? 

THEOREM *(Vitali)* The axiom of choice implies there is a non-Lebesgue-measurable subset of \(\mathbb{R}\). 

    PROOF We just give one of the many possible constructions, this one is due to Vitali. Since \(\mathbb{R}\) and \(\mathbb{Q}\) are both additive groups, consider the quotient group \(\mathbb{R}/\mathbb{Q}\). Using axiom of choice, we can pick an element \(s \in (0,1]\) from each equivalence class in \(\mathbb{R}/\mathbb{Q}\). Let \(S \subset (0,1]\) be the collection of these representatives. Then \(q_1 \neq q_2 \Rightarrow S+q_1 \cap S+q_2 = \emptyset\), and \((0,1] = \bigsqcup_{q\in [0,1)\cap\mathbb{Q}} S+q\). If \(S\) were measurable, we would have \(1 = \mu(S) \times (1+1+...) \), which is not possible. \(\square\) 

\chapter{Measurable Functions} 

A measurable space is a set with a structure, and so it makes sense to ask what a sensible notion of /structure-preserving map/ is. In this case we have a lot of intuition from topology, and so it is straightforward to choose a contravariant definition similar to that of continuous maps.

DEFINITION For measurable spaces \((A,\mathcal{A}), (B,\mathcal{B})\) we say function \(f:A \rightarrow B\) is /measurable/ if \(X\in \mathcal{B}\Rightarrow f^{-1}(X)\in \mathcal{A}\), i.e. if the pre-image of a measurable set is measurable.

    See [[file:./Topology.org::544][this section]] on why the definition of continuous maps must be contravariant. TLDR: you can only /forget information/, so measurable sets cannot come out of nowhere.

\end{document}
